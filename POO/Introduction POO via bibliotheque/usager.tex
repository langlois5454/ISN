\documentclass{article}

  \usepackage[francais]{babel}
	\usepackage[utf8]{inputenc}
  \usepackage[T1]{fontenc}

\begin{document}

En programmation, l'approche objet consiste à penser le système d'information, ou encore le modèle du monde, sous la forme d'un ensemble d'entités. Chaque entité a ses propres caractéristiques et ses propres fonctionnalités. Chaque entité est maîtresse de son évolution au cours du temps.

Dans le TP Bibliothèque, vous avez manipulé trois types d'entités~: les usagers, les livres, et les emprunts.

Prenons l'exemple des usagers. Chaque usager est défini par les mêmes informations (nom, prénom, date de naissance, liste des emprunts), mais chaque usager a ses propres valeurs pour ces informations. En  plus de ces caractéristiques, un usager est détenteur de plusieurs fonctionnalités comme "emprunter un livre".

Nous allons définir une classe des usagers, qui va déterminer ce qu'est un usager type, une sorte de squelette d'usager, une coquille vide. Il faudra aussi définir comment à partir de cette coquille vide, on construit un usager réel.

Voici votre première classe~:

\begin{verbatim}
class Usager:
    def __init__(self,nom,prenom,naissance):
        self.nom = nom
        self.prenom = prenom
        self.date_naissance = naissance
        self.emprunts = []

\end{verbatim}  

Le code Python ci-dessus définit la classe \texttt{Usager}. Cette définition précise qu'un usager est constitué d'un nom (\texttt{nom}), d'un prénom (\texttt{prenom}), d'une date de naissance (\texttt{date\_naissance}) et de la liste de ses emprunts (\texttt{emprunts}). On appellera \textbf{attribut} chacune de ces caractéristiques. Donc, \texttt{nom}, \texttt{prenom}, \texttt{date\_naissance}, \texttt{emprunts} sont les attributs de la classe \texttt{Usager}.

La classe \texttt{Usager} contient une méthode, qui s'appelle \texttt{\_\_init\_\_}. Cette méthode construit un objet de la classe. Tout classe doit contenir une telle méthode. Le terme \textbf{méthode} fait partie du vocabulaire de l'approche objet~:

\begin{itemize}
\item Une \textbf{classe} est constituée d'\textbf{attributs} et de \textbf{méthodes}.
\item Chaque individu représentant de la  classe s'appelle un \textbf{objet}. On dit qu'il s'agit d'une \textbf{instance} de la classe
\item Pour construire un objet de la classe, il faut un \textbf{constructeur}. En Python, il s'agit de la méthode  \texttt{\_\_init\_\_}. 
\end{itemize}

La méthode \texttt{\_\_init\_\_} prend en argument au moins \texttt{self}, qui fait référence à l'objet en cours de construction. Ici, la ligne \texttt{self.nom = nom} se lit~: "nom est un attribut de self, et sa valeur initiale est donnée par le contenu de nom, argument de \texttt{\_\_init\_\_}. Comme on le remarque avec la ligne \texttt{self.date\_naissance = naissance}, le nom de l'attribut et celui de l'argument donnant la valeur initiale ne doivent pas nécessairement être les mêmes, mais c'est mieux ainsi pour la lisibilité du code.

Ici, nous n'avons que que définir \textbf{comment} construire un objet de la classe \texttt{Usager}. Reste encore à le construire. Voici le code permettant cela~:

\begin{verbatim}
   u1 = Usager("Nonyme","Albert","17/09/2000")
\end{verbatim} 

\texttt{u1} est une variable. Pour construire un objet de la classe \texttt{Usager}, on utilise le nom de la classe comme une fonction en passant des valeurs en suivant la signature (liste des arguments) de la méthode \texttt{\_\_init\_\_}. Attention, en Python, cette dernière phrase est fausse. En effet, la méthode \texttt{\_\_init\_\_} de la classe \texttt{Usager} liste 4 arguments (\texttt{self}, \texttt{nom}, \texttt{prenom}, \texttt{naissance}), mais l'appel n'en donne que 3 (\texttt{nom}, \texttt{prenom}, \texttt{naissance}). En fait, le \texttt{self}, à l'appel du constructeur, est implicite~: au moment de la construction, il est ajouté, et fait référence à l'objet en cours de construction.

Maintenant, l'objet \texttt{u1} est construit, et on peut l'utiliser~:

\begin{verbatim}
   print(u1.nom)
   u1.nom = "Toto"
   print(u1.nom)
   u1.emprunts.append("23")
   print(u1.emprunts)
\end{verbatim} 

\paragraph{Exercice 1}~: Téléchargez dans le cours le fichier \texttt{usager.py}. \'Editez-le et ajoutez l'attribut \texttt{date\_renouvellement}, qui est une date qui définit quand l'abonnement de l'usager se termine. Mettez à jour le constructeur (ajout d'un nouvel argument) ainsi que l'appel dans le test au bas du fichier. Testez.


\end{document}