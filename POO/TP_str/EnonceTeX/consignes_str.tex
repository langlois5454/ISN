\documentclass[10pt,a4paper]{sujets-exercices}
\usepackage[utf8]{inputenc}
\usepackage[french]{babel}
\usepackage[T1]{fontenc}
\usepackage[left=2.5cm,right=2.5cm,top=2cm,bottom=2cm]{geometry}

\iutset{%
  type={tp},%
  lieu={ESPE de Lorraine},%
  cours={Formation ISN, niveau 2},%
  annee={2018--2019},%
  titre={Utiliser des objets : la classe \texttt{str}},%
  numero={2}%
}

\renewcommand{\FrenchLabelItem}{\textbullet}

\begin{document}

Dans ce TP, nous allons utiliser une \textbf{classe} disponible nativement dans le langage Python : la classe  \texttt{\textbf{str}} (les chaînes de caractères, ou \emph{string} en anglais).

Une classe correspond à \emph{un type de données évolué} (on parle aussi de type complexe) par opposition aux types primitifs tels que les entiers, réels, booléens, etc.

Pour utiliser une classe (c'est-à-dire un type de donnée complexe), il faut \emph{instancier} (on parle aussi de \emph{construire}, ce qui revient à déclarer et affecter) un élément du type en question. On peut par exemple construire une chaîne de caractères en utilisant les guillemets simples : $$\mbox{\verb!a = 'toto'!}$$
L'instance de la classe (ici dénotée par la variable \verb!a!) est appelée \textbf{objet} (d'où le nom du paradigme de programmation : \emph{Programmation Orientée Objets}).

Les classes définissent non seulement des types de données spécifiques, mais aussi des \textbf{opérations} autorisées sur ces types. Par exemple, il est possible de passer le premier caractère d'une chaîne en majuscule au moyen de la fonction (on parle aussi de \textbf{méthode}) \verb!capitalize!, qui s'utilise comme suit :
$$\mbox{\verb!print('toto'.capitalize())!}$$

Une description (on parle aussi d'\textbf{API} ou interface de programmation d'application, \emph{Application Programming Interface} en anglais) des méthodes disponibles pour la classe \texttt{\textbf{str}} vous est fourni en annexe.
À partir de celle-ci, vous devez répondre aux questions ci-dessous.


\end{document}

