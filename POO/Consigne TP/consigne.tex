\documentclass[10pt,a4paper,onecolumn]{article}
\usepackage[utf8]{inputenc}
\usepackage[french]{babel}
\usepackage[T1]{fontenc}
\usepackage[left=2.5cm,right=2.5cm,top=2cm,bottom=2cm]{geometry}

\renewcommand{\FrenchLabelItem}{\textbullet}

\begin{document}

\title{Système d'information~: bibliothèque}

\author{Langlois David, david.langlois@univ-lorraine.fr}

\maketitle

"Le système d'information (SI) est un ensemble organisé de ressources qui permet de collecter, stocker, traiter et distribuer de l'information" (source~: wikipedia).

Cette activité de programmation va consister à programmer des fonctions de manipulation des données d'une bibliothèque. Le sujet propose une modélisation de l'information sous la forme de listes de listes qu'il faudra manipuler.

\section{Objectifs pédagogiques}

L'objectif de ce TP est triple :

\begin{itemize}
\item Se remémorer tous vos acquis en python via la manipulation de fonctions, de listes de listes, de fichiers texte.
\item Manipuler et modéliser des entités comme des "auteurs", des "livres", des emprunts afin de vous préparer à la structuration d'un programme en objets.
\item Développer le début d'un système d'information de médiathèque afin de vous préparer à la modélisation d'un tel système sous la forme d'une base de données
\end{itemize}

\section{Le système d'information}

Nous allons manipuler les informations nécessaires à une médiathèque :

Un usager est décrit par une liste Python contenant successivement : 

\begin{itemize}
\item Un identifiant (numérique)
\item Un nom (chaîne de caractères)
\item Un prénom (chaîne de caractères)
\item Une date de naissance (une chaîne de caractère de la forme "JJ/MM/AAAA")
\item Les emprunts (une liste d'identifiant d'emprunts, un identifiant étant un entier)
\end{itemize}

Un exemple d'usager~:

\begin{verbatim}
       [1,"Nonyme","Alphonse","01/01/2003",[1,2,4]]
\end{verbatim}


Les usagers sont stockés dans une liste.

Un livre de la bibliothèque est décrit par une liste Python contenant successivement : 

\begin{itemize}
\item Un identifiant (numérique)
\item Un titre (chaîne de caractères)
\item Un auteur (chaîne de caractères)
\item Des mots-clés (une liste de chaînes de caractères
\end{itemize}

Un exemple de livre~:

\begin{verbatim}
       [1,"Nana","Zola Emile",["Drame","Classique","Troisième Empire"]]
\end{verbatim}

Les livres de la bibliothèque sont stockés sous la forme d'une liste.

Un emprunt est décrit par une liste Python contenant successivement :

\begin{itemize}
\item Un identifiant (numérique)
\item L'identifiant du livre emprunté (numérique) ; cet identifiant doit se retrouver dans les liste des livres
\item La date de début d'emprunt (une chaîne de caractère de la forme "JJ/MM/AAAA")
\item La date de retour attendue du livre (une chaîne de caractères de la forme "JJ/MM/AAAA")
\item La date effective de retour du livre (une chaîne de caractères de la forme "JJ/MM/AAAA" si le livre a été rendu, \texttt{None} sinon)
\end{itemize}

Un exemple d'emprunt~:

\begin{verbatim}
       [4,3,"01/01/2015","01/02/2015","15/01/2015"]
\end{verbatim}

Les emprunts de la bibliothèque sont stockés sous la forme d'une liste. L'historique des emprunts est conservé sans nettoyage (certainement pour pouvoir faire des statistiques)~: un livre rendu voit la date de retour effective mise à jour, mais l'emprunt ne disparaît pas de la liste. Le lien entre un emprunt et l'emprunteur se fait par la liste des usagers (rappel~: chaque usager comprend une information sur la liste de ses emprunts)

Les divers identifiants (d'usager, de livre, d'emprunt) permettent d'identifier de manière unique ce à quoi ils se réfèrent, et permettent de faire le lien entre les trois listes. 

\section{Ce qui vous est donné}

Vous avez à disposition un fichier python comprenant l'initialisation d'un système d'information


\section{Fonctions à programmer}




\end{document}

